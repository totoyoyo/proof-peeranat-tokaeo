We study the cardinality of ordinary and disjoint unions of countable sets.

\begin{lem} \label{lem:CardinalityUnions1}
	If $\#A=\#B=\aleph_0$, then $\#(A\times B)=\aleph_0$.
\end{lem}

\hintbox{
	Stitch together some bijections that must exist, to produce a useful bijection.
}

\begin{prop} \label{prop:CardinalityUnions2}
	If, for each $j\in J$, $A_j$ is a nonempty set with cardinality $\#A_j\leq\aleph_0$, and $\#J=\aleph_0$, then
	$$
		\# \bigsqcup_{j\in J} A_j = \aleph_0.
	$$
\end{prop}

\hintbox{
	Start by recalling the definition of disjoint union.
	Find separately an injection and a surjection to the disjoint union of interest, and use the CSB theorem.

	Can you find a surjection from $\NN\times J$ to the disjoint union?

	Note that we asked for all $A_j$ to be nonempty.
	What would happen if all (or all but finitely many) $A_j$ were $\emptyset$?
	This offers a clue to how an injection might be constructed.
}

\begin{prop} \label{prop:CardinalityUnions3}
	If, for each $j\in J$, $\#A_j\leq\aleph_0$, and $\#J=\aleph_0$, then
	$$
		\# \bigcup_{j\in J} A_j \leq \aleph_0.
	$$
\end{prop}

\hintbox{
	Direct construction of a surjection from the disjoint union appearing in proposition~\ref{prop:CardinalityUnions2} to this ordinary union.
}
