%---------DO NOT EDIT THIS INDENTED SECTION
	% Preamble
	\documentclass[11pt,reqno,oneside,a4paper]{article}
	\usepackage[a4paper,includeheadfoot,left=35mm,right=35mm,top=00mm,bottom=30mm,headheight=40mm]{geometry} %sets up the margins
	\input{../texHead-Proof-Standard} % Use the standard texHead for this module. You should not edit this file.
	%%%%%%%%%%%%%%%%%%%%%%%%%%%%%%%%%%%%%%%%%%%%%%%%%%%%%%%%%%%%%%%%%%%%%%%%%%%%%%%%
%
% You can make any edits you like to this file. It is designed as a place
% for you to define your own macros that you want to use across many
% LaTeX documents. You can also define macros in a single document,
% but if you find yourself reusing the same macro in several documents,
% then it probably belongs in here. Look in the "%% Mathmode shortcuts"
% section of texHead-Proof-Standard.tex for some examples of how to define
% your own macros.
%
%%%%%%%%%%%%%%%%%%%%%%%%%%%%%%%%%%%%%%%%%%%%%%%%%%%%%%%%%%%%%%%%%%%%%%%%%%%%%%%%

% My own macros




 % Use your personal additional macros. You may edit this file.
	%---The following code defines the title, author, and date of the document.
	\title{Practice proofs \#4}
	\author{Peeranat (ToTo) Tokaeo}
	\date{\today}   % Using \today automatically updates to the document's build date
%----------------------------------
%---------IF YOU WANT TO DEFINE YOUR OWN MACROS, YOU CAN DO SO EITHER IN ../texHead-Proof-Personal.tex OR FROM HERE ...

%---------... TO HERE
\begin{document}
\maketitle
\thispagestyle{fancy}

%-----------EDIT FROM HERE


We explicitly construct a bijection mapping from the unit interval to the whole real line.

\begin{thm} \label{thm:BijectionRInterval}
	Define $f : (0,1) \to \RR$ by
	$$
		f(x) =\frac{1}{1-x} - \frac{1}{x}.
	$$
	The function $f$ is a bijection.
\end{thm}

\begin{proof}
	%--------------EDIT THIS PART--------------%
	% Prove that $f$ is both a surjection and an injection.
	% It will help to show that $f$ is strictly increasing.
	% The intermediate value theorem is also important; if you have not seen it before, then look it up.
	To prove injectivity, let $0<a<b<1$. Now consider 
	$f(b) - f(a)$.
	\begin{align*}
		f(b) - f(a) &= \frac{1}{1-b} - \frac{1}{b} - \frac{1}{1-a} + \frac{1}{a} \\
		&= \frac{1-a-1+b}{(1-a)(1-b)} + \frac{b-a}{ab} \\
		&= \frac{b-a}{(1-a)(1-b)} + \frac{b-a}{ab}\\
		&= (b-a)\left(\frac{1}{(1-a)(1-b)} + \frac{1}{ab}\right)> 0 
			\end{align*}
	We know the last inequality because $b>a$, so $(b-a)>0$. So we can conclude that $f(b) >f(a)$ for all $a,b$ such that $0<a<b<1$. This means that the function $f$ is strictly increasing, so it will pass the horizontal line test. Hence, it is injective.
	
	To prove surjectivity, let $m$ be a large positive number such that $m>0$. Since $\frac{1}{m}$ and $1-\frac{1}{m}$ are both between 0 and 1, we can consider $f(\frac{1}{m})$ and $f(1-\frac{1}{m})$. So,
	\begin{align*}
	f\left(\frac{1}{m}\right) &= \frac{1}{1-\frac{1}{m}} - \frac{1}		{\frac{1}{m}} \\
	&= \frac{m}{m-1} - m, 
\end{align*}		
and also
	\begin{align*}
	f\left(1-\frac{1}{m}\right) &= \frac{1}{1-1+\frac{1}{m}} - \frac{1}{1-\frac{1}{m}}\\
	&= m - \frac{m}{m-1}.
	\end{align*}
	From this, we know that 
	$$\lim_{m\to\infty} f\left(\frac{1}{m}\right) = \lim_{m\to\infty} \left(\frac{m}{m-1} - m\right) = -\infty,$$ and
	$$\lim_{m\to\infty} f\left(1-\frac{1}{m}\right) = \lim_{m\to\infty} \left(m - \frac{m}{m-1}\right) = \infty.$$
	So for every real number $y$, there exists an $m$ such that $ f\left(\frac{1}{m}\right)<y<f\left(1-\frac{1}{m}\right)$, because the function extends from negative infinity to positive infinity. Because we know that this function $f$ is continuous, we then also know that every real number $y$ is in the range of $f$ by the intermediate value theorem. Hence, the range of $f$ is its codomain, so $f$ is surjective. 
	Because $f$ is injective and surjective, it is bijective.
	%------------------------------------------%
\end{proof}

\end{document}



