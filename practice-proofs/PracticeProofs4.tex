%---------DO NOT EDIT THIS INDENTED SECTION
	% Preamble
	\documentclass[11pt,reqno,oneside,a4paper]{article}
	\usepackage[a4paper,includeheadfoot,left=35mm,right=35mm,top=00mm,bottom=30mm,headheight=40mm]{geometry} %sets up the margins
	%%%%%%%%%%%%%%%%%%%%%%%%%%%%%%%%%%%%%%%%%%%%%%%%%%%%%%%%%%%%%%%%%%%%%%%%%%%%%%%%
%
% This file contains some standard modifications to basic LaTeX2e and
% the article documentclass. DO NOT EDIT THIS FILE, but do look through
% and make use of the shorthands defined herein.
%
%%%%%%%%%%%%%%%%%%%%%%%%%%%%%%%%%%%%%%%%%%%%%%%%%%%%%%%%%%%%%%%%%%%%%%%%%%%%%%%%

% Standard packages
\usepackage{amssymb,amsmath,amsthm}
\usepackage{xcolor,graphicx}
\usepackage{verbatim}
\usepackage{hyperref}
% Layout of headers & footers
\usepackage{titling}
\usepackage{fancyhdr}
\pagestyle{fancy} \lhead{{\theauthor}} \chead{} \rhead{} \lfoot{} \cfoot{\thepage} \rfoot{}

% Hyphenation
\hyphenation{non-zero}

% Instroctor's email address
\newcommand{\InstEmail}{dave.smith@yale-nus.edu.sg}

% Theorem definitions in the amsthm standard
\newtheorem{thm}{Theorem}
\newtheorem{lem}[thm]{Lemma}
\newtheorem{sublem}[thm]{Sublemma}
\newtheorem{prop}[thm]{Proposition}
\newtheorem{cor}[thm]{Corollary}
\newtheorem{conc}[thm]{Conclusion}
\newtheorem{conj}[thm]{Conjecture}
\theoremstyle{definition}
\newtheorem{defn}[thm]{Definition}
\newtheorem{cond}[thm]{Condition}
\newtheorem{asm}[thm]{Assumption}
\newtheorem{ntn}[thm]{Notation}
\newtheorem{prob}[thm]{Problem}
\theoremstyle{remark}
\newtheorem{rmk}[thm]{Remark}
\newtheorem{eg}[thm]{Example}
\newtheorem*{hint}{Hint}

%% Mathmode shortcuts
% Number sets
\newcommand{\NN}{\mathbb N}              % The set of naturals
\newcommand{\NNzero}{\NN^0}              % The set of naturals including zero
\newcommand{\NNone}{\NN}                 % The set of naturals excluding zero
\newcommand{\ZZ}{\mathbb Z}              % The set of integers
\newcommand{\QQ}{\mathbb Q}              % The set of rationals
\newcommand{\RR}{\mathbb R}              % The set of reals
\newcommand{\CC}{\mathbb C}              % The set of complex numbers
\newcommand{\KK}{\mathbb C}              % An arbitrary field
% Modern typesetting for the real and imaginary parts of a complex number
\renewcommand{\Re}{\operatorname*{Re}} \renewcommand{\Im}{\operatorname*{Im}}
% Upright d for derivatives
\newcommand{\D}{\ensuremath{\,\mathrm{d}}}
% Make epsilons look more different from the element symbol
\renewcommand{\epsilon}{\varepsilon}
% Always use slanted forms of \leq, \geq
\renewcommand{\geq}{\geqslant} \renewcommand{\leq}{\leqslant}
% Shorthand for some relations
\newcommand{\po}{\preceq} \newcommand{\rel}{{\mathcal R}} \newcommand{\rels}{\mathbin{\scriptstyle{\mathcal R}}}
% Shorthand for "if and only if" symbol
\newcommand{\Iff}{\ensuremath{\Leftrightarrow}}
% Make bold symbols for vectors
\providecommand{\BVec}[1]{\mathbf{#1}}
% Barred forms of \oplus and \otimes to represent the descents of these binary operators
\newcommand{\oplusbar}{\mathbin{\ooalign{$\hidewidth\overline{\oplus}\hidewidth$\cr$\phantom{\oplus}$}}} \newcommand{\otimesbar}{\mathbin{\ooalign{$\hidewidth\overline{\otimes}\hidewidth$\cr$\phantom{\otimes}$}}}
% Mathematical operators used in Proof
\DeclareMathOperator{\sgn}{sgn}          % The signum of a real number
\DeclareMathOperator{\power}{\mathcal{P}} % The power set of a set
\DeclareMathOperator{\Id}{Id}            % The identity function
\DeclareMathOperator{\Fun}{Fun}          % The set of functions from one set to another
\DeclareMathOperator{\Perm}{Perm}        % The set of permutations on a set
\DeclareMathOperator{\GCD}{GCD}          % The greatest common divisor of two integers
\newcommand{\abs}[1]{\left\lvert#1\right\rvert} % The absolute value of a real number or modulus of a complex number, with automatically scaling delimiters
 % Use the standard texHead for this module. You should not edit this file.
	%%%%%%%%%%%%%%%%%%%%%%%%%%%%%%%%%%%%%%%%%%%%%%%%%%%%%%%%%%%%%%%%%%%%%%%%%%%%%%%%
%
% You can make any edits you like to this file. It is designed as a place
% for you to define your own macros that you want to use across many
% LaTeX documents. You can also define macros in a single document,
% but if you find yourself reusing the same macro in several documents,
% then it probably belongs in here. Look in the "%% Mathmode shortcuts"
% section of texHead-Proof-Standard.tex for some examples of how to define
% your own macros.
%
%%%%%%%%%%%%%%%%%%%%%%%%%%%%%%%%%%%%%%%%%%%%%%%%%%%%%%%%%%%%%%%%%%%%%%%%%%%%%%%%

% My own macros




 % Use your personal additional macros. You may edit this file.
	%---The following code defines the title, author, and date of the document.
	\title{Practice proofs \#4}
	\author{Anonymous}
	\date{\today}   % Using \today automatically updates to the document's build date
%----------------------------------
%---------IF YOU WANT TO DEFINE YOUR OWN MACROS, YOU CAN DO SO EITHER IN ../texHead-Proof-Personal.tex OR FROM HERE ...

%---------... TO HERE
\begin{document}
\maketitle
\thispagestyle{fancy}

%-----------EDIT FROM HERE


We explicitly construct a bijection mapping from the unit interval to the whole real line.

\begin{thm} \label{thm:BijectionRInterval}
	Define $f : (0,1) \to \RR$ by
	$$
		f(x) =\frac{1}{1-x} - \frac{1}{x}.
	$$
	The function $f$ is a bijection.
\end{thm}

\begin{proof}
	%--------------EDIT THIS PART--------------%
	% Prove that $f$ is both a surjection and an injection.
	% It will help to show that $f$ is strictly increasing.
	% The intermediate value theorem is also important; if you have not seen it before, then look it up.
	To prove injectivity, let $0<a<b<1$. Now consider 
	$f(b) - f(a)$.
	\begin{align*}
		f(b) - f(a) &= \frac{1}{1-b} - \frac{1}{b} - \frac{1}{1-a} + \frac{1}{a} \\
		&= \frac{1-a-1+b}{(1-a)(1-b)} + \frac{b-a}{ab} \\
		&= \frac{b-a}{(1-a)(1-b)} + \frac{b-a}{ab}\\
		&= (b-a)\left(\frac{1}{(1-a)(1-b)} + \frac{1}{ab}\right)> 0 
			\end{align*}
	We know the last inequality because $b>a$, so $(b-a)>0$. So we can conclude that $f(b) >f(a)$ for all $a,b$ such that $0<a<b<1$. This means that the function $f$ is strictly increasing, so it will pass the horizontal line test. Hence, it is injective.
	
	To prove surjectivity, let $m$ be positive number such that $m>0$. Since $\frac{1}{m}$ and $1-\frac{1}{m}$ are both between 0 and 1, we can consider $f(\frac{1}{m})$ and $f(1-\frac{1}{m})$. So,
	\begin{align*}
	f\left(\frac{1}{m}\right) &= \frac{1}{1-\frac{1}{m}} - \frac{1}		{\frac{1}{m}} \\
	&= \frac{m}{m-1} - m, 
\end{align*}		
and also
	\begin{align*}
	f\left(1-\frac{1}{m}\right) &= \frac{1}{1-1+\frac{1}{m}} - \frac{1}{1-\frac{1}{m}}\\
	&= m - \frac{m}{m-1}.
	\end{align*}
	From this, we know that 
	$$\lim_{m\to\infty} f\left(\frac{1}{m}\right) = \lim_{m\to\infty} \left(\frac{m}{m-1} - m\right) = -\infty,$$ and
	$$\lim_{m\to\infty} f\left(1-\frac{1}{m}\right) = \lim_{m\to\infty} \left(m - \frac{m}{m-1}\right) = \infty.$$
	So for every real number $y$, there exists an $m$ such that $ f\left(\frac{1}{m}\right)<y<f\left(1-\frac{1}{m}\right)$, because the function extends to negative infinity and to positive infinity. Because we know that this function $f$ is continuous, we then also know that every real number $y$ is in the range of $f$ by the intermediate value theorem. Hence, the range of $f$ is its codomain, so $f$ is surjective. 
	Because $f$ is injective and surjective, it is bijective.
	%------------------------------------------%
\end{proof}

\end{document}



