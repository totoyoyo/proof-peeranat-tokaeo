%---------DO NOT EDIT THIS INDENTED SECTION
	% Preamble
	\documentclass[11pt,reqno,oneside,a4paper]{article}
	\usepackage[a4paper,includeheadfoot,left=35mm,right=35mm,top=00mm,bottom=30mm,headheight=40mm]{geometry} %sets up the margins
	\input{../texHead-Proof-Standard} % Use the standard texHead for this module. You should not edit this file.
	%%%%%%%%%%%%%%%%%%%%%%%%%%%%%%%%%%%%%%%%%%%%%%%%%%%%%%%%%%%%%%%%%%%%%%%%%%%%%%%%
%
% You can make any edits you like to this file. It is designed as a place
% for you to define your own macros that you want to use across many
% LaTeX documents. You can also define macros in a single document,
% but if you find yourself reusing the same macro in several documents,
% then it probably belongs in here. Look in the "%% Mathmode shortcuts"
% section of texHead-Proof-Standard.tex for some examples of how to define
% your own macros.
%
%%%%%%%%%%%%%%%%%%%%%%%%%%%%%%%%%%%%%%%%%%%%%%%%%%%%%%%%%%%%%%%%%%%%%%%%%%%%%%%%

% My own macros




 % Use your personal additional macros. You may edit this file.
	%---The following code defines the title, author, and date of the document.
	\title{Practice proofs \#5}
	\author{Anonymous}
	\date{\today}   % Using \today automatically updates to the document's build date
%----------------------------------
%---------IF YOU WANT TO DEFINE YOUR OWN MACROS, YOU CAN DO SO EITHER IN ../texHead-Proof-Personal.tex OR FROM HERE ...

%---------... TO HERE
\begin{document}
\maketitle
\thispagestyle{fancy}

%-----------EDIT FROM HERE

\begin{abstract}
	%---------DO NOT EDIT THIS FILE

The following exercise is meant to be challenging.
It is expected that your submission will not be perfect, but you will receive feedback on your submission so that you can improve it.
You \textbf{must} complete this exercise entirely on your own.
Do \textbf{not} collaborate with your peers on this exercise.
Do \textbf{not} ask the tutors for help, or search books / web for assistance.
There are hints in the .tex file.
There are more hints in the document \href{run:./PracticeProofsExtraHints.pdf}{PracticeProofsExtraHints.pdf}.
If you need help, please contact Dave by email: \href{mailto:\InstEmail}{\nolinkurl{\InstEmail}}.

You will be graded on participation.
That means it is OK to not have perfect answers, but you will get more out of the exercise if you try your best.

Be sure not to put your name anywhere on the document.
{\color{red}Before submitting, please delete the begin \& end abstract tags and everything between them.}



	
	I fitted the whole model solutions onto $1.5$ pages.
\end{abstract}

We study how cardinality interacts with subsets and power sets.

Recall that we say that two sets are equinumerous if they have the same cardinality.
The following proposition says that the operation ``take the power set'' preserves equinumerosity.
	
\begin{prop} \label{prop:PowerSetsPreserveEquinumerosity}
	Suppose $\# X=\# Y$.
	Then $\# \power(X)=\# \power(Y)$.
	Moreover, if $\# X\leq\# Y$, then $\# \power(X)\leq\# \power(Y)$.
\end{prop}

\begin{proof}
	%--------------EDIT THIS PART--------------%
	% Hint: There must be a bijection $f:X\to Y$.
	% Pick an arbitrary $S\subset X$.
	% Think about the process of taking the image of $S$ in $f$.
	% Is that a function of $S$?
	%------------------------------------------%
\end{proof}

\begin{prop} \label{prop:SubsetCardinality}
	If $X\subset Y$, then $\# X\leq \# Y$.
\end{prop}

\begin{proof}
	%--------------EDIT THIS PART--------------%
	% Hint: Directly construct an injection.
	%------------------------------------------%
\end{proof}

\begin{cor} \label{cor:IntersectionCardinality}
	$\# (X\cap Y) \leq \# X$.
\end{cor}

\begin{proof}
	%--------------EDIT THIS PART--------------%
	% Hint: Use proposition~\ref{prop:SubsetCardinality}.
	%------------------------------------------%
\end{proof}

\end{document}



