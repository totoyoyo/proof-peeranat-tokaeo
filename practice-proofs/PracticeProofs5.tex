%---------DO NOT EDIT THIS INDENTED SECTION
	% Preamble
	\documentclass[11pt,reqno,oneside,a4paper]{article}
	\usepackage[a4paper,includeheadfoot,left=35mm,right=35mm,top=00mm,bottom=30mm,headheight=40mm]{geometry} %sets up the margins
	\input{../texHead-Proof-Standard} % Use the standard texHead for this module. You should not edit this file.
	%%%%%%%%%%%%%%%%%%%%%%%%%%%%%%%%%%%%%%%%%%%%%%%%%%%%%%%%%%%%%%%%%%%%%%%%%%%%%%%%
%
% You can make any edits you like to this file. It is designed as a place
% for you to define your own macros that you want to use across many
% LaTeX documents. You can also define macros in a single document,
% but if you find yourself reusing the same macro in several documents,
% then it probably belongs in here. Look in the "%% Mathmode shortcuts"
% section of texHead-Proof-Standard.tex for some examples of how to define
% your own macros.
%
%%%%%%%%%%%%%%%%%%%%%%%%%%%%%%%%%%%%%%%%%%%%%%%%%%%%%%%%%%%%%%%%%%%%%%%%%%%%%%%%

% My own macros




 % Use your personal additional macros. You may edit this file.
	%---The following code defines the title, author, and date of the document.
	\title{Practice proofs \#5}
	\author{Anonymous}
	\date{\today}   % Using \today automatically updates to the document's build date
%----------------------------------
%---------IF YOU WANT TO DEFINE YOUR OWN MACROS, YOU CAN DO SO EITHER IN ../texHead-Proof-Personal.tex OR FROM HERE ...

%---------... TO HERE
\begin{document}
\maketitle
\thispagestyle{fancy}

%-----------EDIT FROM HERE

We study how cardinality interacts with subsets and power sets.

Recall that we say that two sets are equinumerous if they have the same cardinality.
The following proposition says that the operation ``take the power set'' preserves equinumerosity.
	
\begin{prop} \label{prop:PowerSetsPreserveEquinumerosity}
	Suppose $\# X=\# Y$.
	Then $\# \power(X)=\# \power(Y)$.
	Moreover, if $\# X\leq\# Y$, then $\# \power(X)\leq\# \power(Y)$.
\end{prop}

\begin{proof}
	%--------------EDIT THIS PART--------------%
	% Hint: There must be a bijection $f:X\to Y$.
	% Pick an arbitrary $S\subset X$.
	% Think about the process of taking the image of $S$ in $f$.
	% Is that a function of $S$?
	If $\# X=\# Y$, then there exists a bijection $f:X\to Y$.	
	Let us define a function $g: \power(X) \to \power(Y)$ by $g(S) = \{f(x): x \in S\}$.
	Suppose $S, T \subset X$ such that $S \neq T$. 
	If $g(S)=g(T)$, then $\forall x\in S$, $f(x) \in g(S)$. 
	But $g(S)=g(T)$, so $f(x) \in g(T)$ also. 
	This implies that $\forall x\in S$, $\exists y\in T$ such that $f(x)=f(y)$. If we know that $f$ is injective, then $x=y$. This means that $\forall x\in S,$  $x\in T \implies S \subset T$.
	
	To show that $T \subset S$, we make the same assumptions. If $g(T)=g(S)$, then $\forall x\in T$, $f(x) \in g(T)$. 
	But $g(T)=g(S)$ so $f(x), \in g(S)$ also. 
	This implies that $\forall x\in T$, $\exists y\in S$ such that $f(x)=f(y)$. If we know that $f$ is injective, then $x=y$. This means that $\forall x\in T,$  $x\in S \implies T \subset S$.
	
	So we have shown that, given that $f$ is injective, $g(S)=g(T)$ implies $S=T$, which means that $g$ is an injection. So, the second part of the theorem is proved.
	
	To show that $g$ is also surjective and hence a bijection, let $T$ be an arbitrary element of $\power(Y)$, that is $T \subset Y$. To show surjectivity, we just need to find $S \in \power(X)$ such that $g(S)=T$. Let us define $S=\{f^{-1}(y):y\in T\}$. The existence of $f^{-1}: Y \to X$ is guaranteed because we know that $f$ is a bijection. Applying $g$ to this defined $S$ yields,
	\begin{align*}
		g(S) &= \{f(f^{-1}(y)):y\in T\}\\
		&= \{y:y\in T\} \\
		&= T
	\end{align*}
	So $g$ is surjective and injective, and therefore $\# \power(X)=\# \power(Y)$.
	%------------------------------------------%
\end{proof}

\begin{prop} \label{prop:SubsetCardinality}
	If $X\subset Y$, then $\# X\leq \# Y$.
\end{prop}

\begin{proof}
	%--------------EDIT THIS PART--------------%
	Suppose $X\subset Y$ and $x \in X$. Let us define a function $f: X\to Y$ in the following way: $f(x) = x$. Since each $x$ is in $X$, each $x$ is also in $Y$.
	
	To show that $f$ is injective, suppose there exists $x_1,x_2 \in X$ such that $x_1 \neq x_2$. If $f(x_1) = f(x_2)$, then $x_1=f(x_1) = f(x_2)=x_2$, which is a contradiction to our assumption. Hence, $f(x_1) = f(x_2) \implies x_1 = x_2$ and $f$ is injective. So, by definition of injectivity, $\# X\leq \# Y$.
	% Hint: Directly construct an injection.
	%------------------------------------------%
\end{proof}

\begin{cor} \label{cor:IntersectionCardinality}
	$\# (X\cap Y) \leq \# X$.
\end{cor}

\begin{proof}
	%--------------EDIT THIS PART--------------%
	% Hint: Use proposition~\ref{prop:SubsetCardinality}.
	Because $(X\cap Y)$ is a subset of $X$, Proposition 2 tells us that $\# (X\cap Y) \leq \# X$.
	%------------------------------------------%
\end{proof}

\end{document}



