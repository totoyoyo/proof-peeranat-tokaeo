%---------DO NOT EDIT THIS INDENTED SECTION
	% Preamble
	\documentclass[11pt,reqno,oneside,a4paper]{article}
	\usepackage[a4paper,includeheadfoot,left=35mm,right=35mm,top=00mm,bottom=30mm,headheight=40mm]{geometry} %sets up the margins
	\input{../texHead-Proof-Standard} % Use the standard texHead for this module. You should not edit this file.
	%%%%%%%%%%%%%%%%%%%%%%%%%%%%%%%%%%%%%%%%%%%%%%%%%%%%%%%%%%%%%%%%%%%%%%%%%%%%%%%%
%
% You can make any edits you like to this file. It is designed as a place
% for you to define your own macros that you want to use across many
% LaTeX documents. You can also define macros in a single document,
% but if you find yourself reusing the same macro in several documents,
% then it probably belongs in here. Look in the "%% Mathmode shortcuts"
% section of texHead-Proof-Standard.tex for some examples of how to define
% your own macros.
%
%%%%%%%%%%%%%%%%%%%%%%%%%%%%%%%%%%%%%%%%%%%%%%%%%%%%%%%%%%%%%%%%%%%%%%%%%%%%%%%%

% My own macros




 % Use your personal additional macros. You may edit this file.
	%---The following code defines the title, author, and date of the document.
	\title{Practice proofs \#6}
	\author{Anonymous}
	\date{\today}   % Using \today automatically updates to the document's build date
%----------------------------------
%---------IF YOU WANT TO DEFINE YOUR OWN MACROS, YOU CAN DO SO EITHER IN ../texHead-Proof-Personal.tex OR FROM HERE ...

%---------... TO HERE
\begin{document}
\maketitle
\thispagestyle{fancy}

%-----------EDIT FROM HERE

We study the cardinality of ordinary and disjoint unions of countable sets.

\begin{lem} \label{lem:CardinalityUnions1}
	If $\#A=\#B=\aleph_0$, then $\#(A\times B)=\aleph_0$.
\end{lem}

\begin{proof}
	%--------------EDIT THIS PART--------------%
	% Hint: Direct proof.
	We know that $\#A=\#B=\aleph_0=\#\NN$. So $\#(A\times B) = \#A \times \#B = \#\NN \times \#\NN = \#(\NN \times \NN) = \aleph_0$. Hence, $\#(A\times B)=\aleph_0$.
	%------------------------------------------%
\end{proof}

\begin{prop} \label{prop:CardinalityUnions2}
	If, for each $j\in J$, $A_j$ is a nonempty set with cardinality $\#A_j\leq\aleph_0$, and $\#J=\aleph_0$, then
	$$
		\# \bigsqcup_{j\in J} A_j = \aleph_0.
	$$
\end{prop}

\begin{proof}
	%--------------EDIT THIS PART--------------%
	% Hint: Start by recalling the definition of disjoint union.
	% Find separately an injection and a surjection to the disjoint union of interest, and use the CSB theorem.
	To show that an injection exists, consider the set $J$. We can easily construct an injection $J \to \bigsqcup_{j\in J} A_j$ where each element in $J$ is mapped to one element in $\bigsqcup_{j\in J} A_j$ that has the second term equal to that $J$. 
	It is guaranteed that at least one such element exists because $A_j$ is a nonempty set for all $j$. Because $\aleph_0 = \#J \leq \# \bigsqcup_{j\in J} A_j$, we know $\aleph_0 \leq \# \bigsqcup_{j\in J} A_j$.

	Now to show that there is a surjection from the naturals to $\bigsqcup_{j\in J} A_j$, consider the set $\NN \times J$. 
	We can map each element of $\NN \times J$ to $\bigsqcup_{j\in J} A_j$ that shares the same second element. 
	For every specific $j \in J$, there are more or equal number of elements that has $j$ as its second term in $\NN \times J$ than in $\bigsqcup_{j\in J} A_j$, because $\aleph_0\geq\#A_j$. Therefore, there certainly exists a surjection from $\NN \times J$ to $\bigsqcup_{j\in J} A_j$. This means that $\aleph_0 = \# \NN \times J \geq \# \bigsqcup_{j\in J} A_j$ and also from injection that $\aleph_0 \leq \# \bigsqcup_{j\in J} A_j$.
	
	So, $\# \bigsqcup_{j\in J} A_j = \aleph_0$.
	%------------------------------------------%
\end{proof}

\begin{prop} \label{prop:CardinalityUnions3}
	If, for each $j\in J$, $\#A_j\leq\aleph_0$, and $\#J=\aleph_0$, then
	$$
		\# \bigcup_{j\in J} A_j \leq \aleph_0.
	$$
\end{prop}

\begin{proof}
	%--------------EDIT THIS PART--------------%
	% Hint: Direct construction.
	Consider the disjoint union $\bigsqcup_{j\in J} A_j$. We can define a function $f: \bigsqcup_{j\in J} A_j \to \bigcup_{j\in J} A_j$ such that $f(a,j) = a$ where $a \in \bigcup_{j\in J} A_j$. 
	This function is clearly surjective because every element in the union is also an element in the disjoint union, if we remove the second index term. So, we have defined $f$ to essentially remove the second index term of each element of $\bigsqcup_{j\in J} A_j$. 
	
	So, $\aleph_0 = \#\bigsqcup_{j\in J} A_j \geq \# \bigcup_{j\in J} A_j$, which implies $\# \bigcup_{j\in J} A_j \leq \aleph_0$.
	%------------------------------------------%
\end{proof}

\end{document}



